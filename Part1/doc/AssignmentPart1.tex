\documentclass[11pt]{article}
\usepackage{a4wide}
\usepackage{latexsym}
\usepackage{amssymb}
\usepackage{epic}
\usepackage{graphicx}
%\pagestyle{empty}
\newcommand{\tr}{\mbox{\sf true}}
\newcommand{\fa}{\mbox{\sf false}}
\newcommand{\bimp}{\leftrightarrow}


\begin{document}

\section*{
\begin{center}
Practical Assignment Part 1\\
Automated Reasoning 2IMF25\\
\end{center}
}

\begin{center}
Technische Universiteit Eindhoven\\
Jiahuan Zhang (j.4.zhang@student.tue.nl)\\
Hector Joao Rivera Verduzco (h.j.rivera.verduzco@student.tue.nl)\\

\end{center}
\date{\today}
\vspace{5mm}

\section*{Problem 1}

Six trucks have to deliver pallets of obscure building blocks to a magic factory. Every truck has a capacity of 7800 kg and can carry at most eight pallets. In total, the following has to be delivered:
\begin{itemize}
  \item Four pallets of nuzzles, each of weight 700 kg.
  \item A number of pallets of prittles, each of weight 800 kg.
  \item Eight pallets of skipples, each of weight 1000 kg.
  \item Ten pallets of crottles, each of weight 1500 kg.
  \item Five pallets of dupples, each of weight 100 kg.
\end{itemize}
Prittles and crottles are an explosive combination: they are not allowed to be put in the same truck.\\
Skipples need to be cooled; only two of the six trucks have facility for cooling skipples.\\
Dupples are very valuable; to distribute the risk of loss no two pallets of dupples may be in the same truck.\\
Investigate what is the maximum number of pallets of prittles that can be delivered, and show how for that number all pallets may be divided over the six trucks.

\vspace{4mm}

\subsection*{Solution:}

We generalize the problem to offer $m$ trucks to deliver the pallets. $m$ is a positive integer. Then we introduce some variables $t_{ij}$ for $i = 1, ..., m$ and $j = 1,...,5$, which represents the number of pallets of obscure building blocks $j$ on the $i$-th truck. $t_{ij}$ is a nature number. Every truck has a capacity, which is denoted as $C$, and the maximum number of pallets each truck can carry is $M$ kg.

The pallets of the same building blocks have the same weight, $weight(j)$. Each kind of building blocks has a predefined number, $\sharp pallets(j)$.

Now we consider the conditions for the delivery.

\begin{description}
  \item[Condition 1:] $t_{ij}$ should be no less than $0$.

  This is expressed by the formula
  \[ \bigwedge_{i,j:1 \leq i \leq m \wedge 1 \leq j \leq 5} t_{ij} \geq 0.\]
  \item[Condition 2:] Every truck has a capacity of $C$ kg.

  \[ \bigwedge_{i=1}^m  (\sum_{j=1}^5 t_{ij} \times weight(j)) \leq C .\]
  \item[Condition 3:] Every truck can carry at most eight pallets.

  \[ \bigwedge_{i=1}^m (\sum_{j=1}^5 t_{ij}) \leq M .\]
  \item[Condition 4:] The total numbers of the pallets of the other obscure building blocks should be exact the same as the given number.

  \[ \bigwedge_{1\leq j \leq 5} (\sum_{i=1}^m t_{ij}) = \sharp pallets(j) .\]

  \item[Condition 5:] Prittles and crottles are not allowed to be put in the same truck.

  \[ \bigwedge_{i=1}^m t_{i2}=0 \vee t_{i4}=0 .\]
  \item[Condition 6:] Only two of the trucks can deliver skipples.

      So let's assume the $m$-th and $m-2$-th trucks can deliver skipples.

  \[ t_{m3} + t_{(m-2)3} = M .\]
  \item[Condition 7:] No two pallets of dupples may be in the same truck.

  \[ \bigwedge_{i=1}^m t_{i5} \leq 1 .\]
\end{description}

The total formula now consists of the conjunction of all these
ingredients, that is,
\[ \bigwedge_{i,j:1 \leq i \leq m, 1 \leq j \leq 5} t_{ij} \geq 0 \;\; \wedge \]
\[ \bigwedge_{i=1}^m  (\sum_{j=1}^5 t_{ij} \times weight(j)) \leq C \;\; \wedge \]
\[ \bigwedge_{i=1}^m (\sum_{j=1}^5 t_{ij}) \leq M \;\; \wedge \]
\[ \bigwedge_{1\leq j \leq5, j\neq2} (\sum_{i=1}^m t_{ij}) = \sharp pallets(j) \;\; \wedge \]
\[ \bigwedge_{i=1}^m t_{i2}=0 \vee t_{i4}=0 \;\; \wedge \]
\[ t_{m3} + t_{(m-1)3} = M \;\; \wedge \]
\[ \bigwedge_{i=1}^m t_{i5} \leq 1 \]

This formula is easily expressed in SMT syntax. For Problem 1, we define $m = 6$, $M = 8$, $C = 7800$, and the table below.
\begin{center}
\begin{tabular}{|c|c|c|c|c|c|}
  \hline
  % after \\: \hline or \cline{col1-col2} \cline{col3-col4} ...
    & nuzzles & prittles & skipples & crottles & dupples \\
  $j$ & 1 & 2 & 3 & 4 & 5 \\
  $weight(j)$ & 700kg & 800kg & 1000kg & 1500kg & 100kg \\
  $\sharp pallets(j)$ & 4 & $n$ & 8 & 10 & 5 \\
  \hline
\end{tabular}
\end{center}

With the values above, we get the following Yices codes.

{\footnotesize

{\tt (benchmark Part1\_1.smt}

{\tt :logic $QF\_ALIA$}

{\tt :extrafuns (}

{\tt (t11 Int) (t12 Int) (t13 Int) (t14 Int) (t15 Int) }

{\tt (t21 Int) (t22 Int) (t23 Int) (t24 Int) (t25 Int) }

{\tt (t31 Int) (t32 Int) (t33 Int) (t34 Int) (t35 Int) }

{\tt (t41 Int) (t42 Int) (t43 Int) (t44 Int) (t45 Int) }

{\tt (t51 Int) (t52 Int) (t53 Int) (t54 Int) (t55 Int) }

{\tt (t61 Int) (t62 Int) (t63 Int) (t64 Int) (t65 Int) }

{\tt )}

{\tt :formula}

{\tt   (and}

{\tt (>= t11 0) (>= t12 0) (>= t13 0) (>= t14 0) (>= t15 0)}

{\tt (>= t21 0) (>= t22 0) (>= t23 0) (>= t24 0) (>= t25 0)}

$\cdots \cdots$

{\tt (>= t61 0) (>= t62 0) (>= t63 0) (>= t64 0) (>= t65 0) }

{\tt (<= (+ (* t11 700) (* t12 800) (* t13 1000) (* t14 1500) (* t15 100)) 7800)}

{\tt (<= (+ (* t21 700) (* t22 800) (* t23 1000) (* t24 1500) (* t25 100)) 7800)}

{\tt (<= (+ (* t31 700) (* t32 800) (* t33 1000) (* t34 1500) (* t35 100)) 7800)}

{\tt (<= (+ (* t41 700) (* t42 800) (* t43 1000) (* t44 1500) (* t45 100)) 7800)}

{\tt (<= (+ (* t51 700) (* t52 800) (* t53 1000) (* t54 1500) (* t55 100)) 7800)}

{\tt (<= (+ (* t61 700) (* t62 800) (* t63 1000) (* t64 1500) (* t65 100)) 7800) }

{\tt (<= (+ t11 t12 t13 t14 t15) 8)}

{\tt (<= (+ t21 t22 t23 t24 t25) 8)}

{\tt (<= (+ t31 t32 t33 t34 t35) 8)}

{\tt (<= (+ t41 t42 t43 t44 t45) 8)}

{\tt (<= (+ t51 t52 t53 t54 t55) 8)}

{\tt (<= (+ t61 t62 t63 t64 t65) 8)}

{\tt (= (+ t11 t21 t31 t41 t51 t61) 4)}

{\tt (= (+ t12 t22 t32 t42 t52 t62) 18)}

{\tt (= (+ t13 t23 t33 t43 t53 t63) 8)}

{\tt (= (+ t14 t24 t34 t44 t54 t64) 10)}

{\tt (= (+ t15 t25 t35 t45 t55 t65) 5)}

{\tt (or (= t12 0) (= t14 0)) }

{\tt (or (= t22 0) (= t24 0)) }

{\tt (or (= t32 0) (= t34 0)) }

{\tt (or (= t42 0) (= t44 0)) }

{\tt (or (= t52 0) (= t54 0)) }

{\tt (or (= t62 0) (= t64 0)) }

{\tt (or}

{\tt (= (+ t43 t63) 8) }

{\tt (<= t15 1) (<= t25 1) (<= t35 1) (<= t45 1) (<= t55 1) (<= t65 1)}

{\tt )) }
}

Applying {\tt yices-smt -m Part1\_1.smt} several times, we find when the number of pallets of prittles is 19, it is UNSAT. When the number is 18, it is SAT. Therefore, we conclude that the maximal number of pallets of prittles is 18.
The following result is yielded within a fraction of a second:

{\footnotesize

{\tt sat}

{\tt (= t52 7)}

{\tt (= t11 0)}

{\tt (= t14 5)}

{\tt (= t34 0)}

{\tt (= t21 4)}

{\tt (= t15 1)}

{\tt (= t44 2)}

{\tt (= t45 1)}

{\tt (= t54 0)}

{\tt (= t33 0)}

{\tt (= t35 0)}

{\tt (= t51 0)}

{\tt (= t23 0)}

{\tt (= t53 0)}

{\tt (= t65 1)}

{\tt (= t13 0)}

{\tt (= t62 0)}

{\tt (= t25 1)}

{\tt (= t32 8)}

{\tt (= t41 0)}

{\tt (= t63 4)}

{\tt (= t43 4)}

{\tt (= t42 0)}

{\tt (= t12 0)}

{\tt (= t64 0)}

{\tt (= t55 1)}

{\tt (= t22 0)}

{\tt (= t24 3)}

{\tt (= t61 0)}

{\tt (= t31 0)}

}

\begin{table}
  \centering
  \begin{tabular}{|l|c|c|c|c|c|}
    \hline
    % after \\: \hline or \cline{col1-col2} \cline{col3-col4} ...
     & nuzzles & prittles & skipples & crottles & dupples \\
    Truck 1 & 0 & 0 & 0 & 5 & 1 \\
    Truck 2 & 4 & 0 & 0 & 3 & 1 \\
    Truck 3 & 0 & 8 & 0 & 0 & 0 \\
    Truck 4 & 0 & 0 & 4 & 2 & 1 \\
    Truck 5 & 0 & 7 & 0 & 0 & 1 \\
    Truck 6 & 0 & 3 & 4 & 0 & 1 \\
    \hline
  \end{tabular}
\end{table}

\subsection*{Remark}
In this problem, we are required to find out the maximum number of the pallets of prittles. Such kind of maximum values searching is likely to become very time-consuming since there is no explicitly close upper bound to start the searching.

For this problem, we first find out the theoretically possible maximum number of the pallets of prittles through calculation. Because the capacity of each track is given, and it is also provided the total weights of the pallets of the other obscure building blocks, we can estimate the number.

  $\frac{7800 \times 6 - 700\times4 - 1000\times8 - 1500\times10 - 100\times5}{800} = 25.625$

Because the number of pallets is a nature number, the maximum number should not be more than 25. Therefore, we can start the debugging from $n = 25$ downwards.

\subsection*{Generalization}

We generalized all the invariants except the total number of kinds of the building blocks in this problem, because most of the requirements come from the features of different building blocks, which indicates that adding more different building blocks will generate more requirements, and as a consequent, the entire formula we summarized in the end of the solution is not applicable.

One more note is for the people who are interested in setting the number of trucks, $m$, larger than $10$. They need to care care about the notations of the variables $t_{ij}$. For instance, the number of pallets of building blocks labeled $1$ on the eleventh truck, $t_{11,1}$, is expressed as  $t111$ in Yices codes. This expression can also represent the number of pallets of building blocks labeled $11$ on the first truck, $t_{1,11}$. An extra symbol between the two numbers $i$ and $j$ is required to avoid this ambiguity.

Decreasing the number of trucks is more of interest. We did some testing to figure out how many pallets of prittles can satisfy these conditions with fewer trucks. When there are five trucks in all, ten pallets of prittles can be delivered with the satisfiability of the conditions. With fewer trucks, no satisfiability is reached. Since one truck can only deliver one pallet of drupples due to the condition, the number of trucks has to be larger than the number of pallets of drupples for delivery. It can be expressed as this
\[ \sum_{i=1}^m t_{i5}\leq m \]


\section*{Problem 2}

Give a chip design containing three power components and eight regular components satisfying the following constraints:
\begin{itemize}
  \item The width of the chip is 29 and the height is 22.
  \item All power components have width 4 and height 2.
  \item The sizes of the eight regular components are $9 \times 7$, $12 \times 6$, $10 \times 7$, $18 \times 5$, $20 \times 4$, $10 \times 6$, $8 \times 6$ and $10 \times 8$, respectively.
  \item All components may be turned 90, but may not overlap.
  \item In order to get power, all regular components should directly be connected to a power component, that is, an edge of the component should have at least one point in common with an edge of the power component.
  \item Due to limits on heat production the power components should be not too close: their centres should differ at least 17 in either the $x$ direction or the $y$ direction (or both).
\end{itemize}

\vspace{4mm}

\subsection*{Solution:}
First of all, we assume the chip plan is a positive plane whose left-bottom corner has coordinate $(0, 0)$, and right-top corner has coordinate $(29, 22)$.
Next, we introduce coordinates to each corner of each components $(x_{ij}, y_{ij})$ to specify them.
See below.

\begin{enumerate}
  \item For the three power components:\\
  Power component 1 has coordinates $(x_{11}, y_{11})$, $(x_{12}, y_{11})$, $(x_{12}, y_{11})$ and $(x_{12}, y_{12})$.\\
  Power component 2 has coordinates $(x_{21}, y_{21})$, $(x_{22}, y_{21})$, $(x_{22}, y_{21})$ and $(x_{22}, y_{22})$.\\
  Power component 3 has coordinates $(x_{31}, y_{31})$, $(x_{32}, y_{31})$, $(x_{32}, y_{31})$ and $(x_{32}, y_{32})$.
  \item For the eight regular components:\\
  Regular component 1 has coordinates $(n_{11}, m_{11})$, $(n_{12}, m_{11})$, $(n_{12}, m_{11})$ and $(n_{12}, m_{12})$.\\
  Regular component 2 has coordinates $(n_{21}, m_{21})$, $(n_{22}, m_{21})$, $(n_{22}, m_{21})$ and $(n_{22}, m_{22})$.

  $\cdots \cdots$

  Regular component 8 has coordinates $(n_{81}, m_{81})$, $(n_{82}, m_{81})$, $(n_{82}, m_{81})$ and $(n_{82}, m_{82})$.
\end{enumerate}
Thus, we plan to put the components onto the plan in terms of the constraints.

\begin{description}
  \item[Constraint 1:] Now the chip is a positive plane, so all the coordinates should be positive. The width of the chip is 29 and the height is 22.

  Then we have
  \[  \bigwedge_{i=1}^3 29 \geq x_{i2} \geq x_{i1} \geq 0 \wedge 22 \geq y_{i2} \geq y_{i1} \geq 0 \;\; \wedge \]
  \[  \bigwedge_{i=1}^8 29 \geq n_{i2} \geq n_{i1} \geq 0 \wedge 22 \geq m_{i2} \geq m_{i1} \geq 0 \]
  \item[Constraint 2:] All power components have width 4 and height 2,
  and the sizes of the eight regular components are $9 \times 7$, $12 \times 6$, $10 \times 7$, $18 \times 5$, $20 \times 4$, $10 \times 6$, $8 \times 6$ and $10 \times 8$, respectively.
  All components may be turned 90.

  Then we have
  \[  \bigwedge_{i=1}^3 ((x_{i2} - x_{i1} = 4 \; \wedge \; y_{i2} - y_{i1} = 2) \;\; \vee \]
  \[ (x_{i2} - x_{i1} = 2 \; \wedge \; y_{i2} - y_{i1} = 4)) \]
  So do the coordinates for the Regular components. However their sizes are distinct,
  this makes more writing and coding. Here we only show the clause for the first regular component,
  whose size is $9 \times 7$. \\
  \[ (n_{12} - n_{11} = 9 \; \wedge \; m_{12} - m_{11} = 7) \;\; \vee (n_{12} - n_{11} = 7 \; \wedge \; m_{12} - m_{11} = 9) \]
  \item[Constraint 3:] All components may not overlap.

\begin{center}
\includegraphics[width=0.5\textwidth]{Part1_2_1.png}
\end{center}
  As the figure shows, only if a component's $x_{i1}$ is larger than the other one's $x_{k2}$, then these two components are not overlapped. We also have the case in the y direction.

  Among the power components:

  \[  \bigwedge_{i,k: 1 \leq i \leq 3, 1 \leq k \leq 3, i \neq k}
  x_{i2} \leq x_{k1} \; \vee \; x_{k2} \leq x_{i1} \; \vee \; y_{i2} \leq y_{k1} \; \vee \; y_{k2} \leq y_{i1} \]

  Among the regular components:

  \[  \bigwedge_{i,k: 1 \leq i \leq 8, 1 \leq k \leq 8, i \neq k}
  n_{i2} \leq n_{k1} \; \vee \; n_{k2} \leq n_{i1} \; \vee \; m_{i2} \leq m_{k1} \; \vee \; m_{k2} \leq m_{i1}  \]

  Between the power components and the regular components:

  \[  \bigwedge_{i,k: 1 \leq i \leq 3, 1 \leq k \leq 8, i \neq k}
   x_{i2} \leq n_{k1} \; \vee \; n_{k2} \leq x_{i1} \; \vee \; y_{i2} \leq m_{k1} \; \vee \; m_{k2} \leq y_{i1} \]

  \item[Constraint 4:] An edge of the regular component should have at least one point in common with an edge of the power component.

\begin{center}
\includegraphics[width=0.5\textwidth]{Part1_2_2.png}
\end{center}

  As the figure above shows, if two components are touched at least one point, then at least one coordinate of a component is the same as the other one. We consider it from the inverse way to make the clauses simpler. If two components have the same $x_{i1}$ value, only if their $y$-direction sides are not touched, then these two components are not in common with an edge of each other.
  We denote this condition as follows.

  \[  \bigwedge_{i,k: 1 \leq i \leq 3, 1 \leq k \leq 8}
  ((x_{i2} = n_{k1} \; \vee \; x_{i1} = n_{k2}) \; \wedge \;
  \neg (y_{i2} < m_{k1} \vee m_{k2} < y_{i1}) \; \vee \; \]
  \[  ((y_{i2} = m_{k1} \; \vee \; y_{i1} = m_{k2}) \; \wedge \;
  \neg (x_{i2} < n_{k1} \vee n_{k2} < x_{i1}))) \]
  \item[Constraint 5:] The power components' centres should differ at least 17 in either the $x$ direction or the $y$ direction (or both).

  The centers of the power components are $(x_{i1} + \frac{x_{i2} - x_{i1}}{2}, y_{i1} + \frac{y_{i2} - y_{i1}}{2} )$.

  For each pair of the power components, we have

  \[  \bigwedge_{i,k: 1 \leq i < k \leq 3}
  [ (x_{i1} + \frac{x_{i2} - x_{i1}}{2}) - (x_{k1} + \frac{x_{k2} - x_{k1}}{2}) \geq 17 \; \vee \; \]
  \[ (x_{k1} + \frac{x_{k2} - x_{k1}}{2}) - (x_{i1} + \frac{x_{i2} - x_{i1}}{2}) \geq 17 \; \vee \; \]
  \[ (y_{k1} + \frac{y_{k2} - y_{k1}}{2}) - (y_{i1} + \frac{y_{i2} - y_{i1}}{2}) \geq 17 \; \vee \; \]
  \[ (y_{i1} + \frac{y_{i2} - y_{i1}}{2}) - (y_{k1} + \frac{y_{k2} - y_{k1}}{2}) \geq 17 ] \]


\end{description}

Making a conjunction of all the clauses derived from the constraints, we made the Yices smt cods.

This formula is easily expressed in SMT syntax.

{\footnotesize

{\tt (benchmark Part1\_2.smt}

{\tt :logic $QF\_LIA$}

{\tt :extrafuns (}

{\tt (x11 Int) (x12 Int) (y11 Int) (y12 Int)}

{\tt (x21 Int) (x22 Int) (y21 Int) (y22 Int)}

{\tt (x31 Int) (x32 Int) (y31 Int) (y32 Int)}

{\tt (n11 Int) (n12 Int) (m11 Int) (m12 Int)}

{\tt (n21 Int) (n22 Int) (m21 Int) (m22 Int)}

$\cdots \cdots$

{\tt (n81 Int) (n82 Int) (m81 Int) (m82 Int)}

{\tt )}

{\tt :formula (and}

{\tt (>= 29 x12) (>= x12 x11) (>= x11 0) (>= 22 y12) (>= y12 y11) (>= y11 0)}

{\tt (>= 29 x22) (>= x22 x21) (>= x21 0) (>= 22 y22) (>= y22 y21) (>= y21 0)}

{\tt (>= 29 x32) (>= x32 x31) (>= x31 0) (>= 22 y32) (>= y32 y31) (>= y31 0)}

{\tt (>= 29 n12) (>= n12 n11) (>= n11 0) (>= 22 m12) (>= m12 m11) (>= m11 0)}

{\tt (>= 29 n22) (>= n22 n21) (>= n21 0) (>= 22 m22) (>= m22 m21) (>= m21 0)}

$\cdots \cdots$

{\tt (>= 29 n82) (>= n82 n81) (>= n81 0) (>= 22 m82) (>= m82 m81) (>= m81 0)}

{\tt (or (and (>= 29 x12) (>= 22 y12) (= (- x12 x11) 4) (= (- y12 y11) 2))}

{\tt (and (>= 22 x12) (>= 29 y12) (= (- y12 y11) 4) (= (- x12 x11) 2)))}

$\cdots \cdots$

{\tt (or (and (= (- x32 x31) 4) (= (- y32 y31) 2)) (and (= (- x32 x31) 2) (= (- y32 y31) 4)))}

{\tt (or (and (= (- n12 n11) 9) (= (- m12 m11) 7)) (and (= (- n12 n11) 7) (= (- m12 m11) 9)))}

$\cdots \cdots$

{\tt (or (and (= (- n82 n81) 10) (= (- m82 m81) 8)) (and (= (- n82 n81) 8) (= (- m82 m81) 10)))}

{\tt ;; not overlap among power components}

{\tt (or (<= x12 x21) (<= x22 x11) (<= y12 y21) (<= y22 y11))}

{\tt (or (<= x12 x31) (<= x32 x11) (<= y12 y31) (<= y32 y11))}

{\tt (or (<= x32 x21) (<= x22 x31) (<= y32 y21) (<= y22 y31))}

{\tt ;; not overlap among regular components}

{\tt (or (<= n12 n21) (<= n22 n11) (<= m12 m21) (<= m22 m11))}

{\tt (or (<= n12 n31) (<= n32 n11) (<= m12 m31) (<= m32 m11))}

$\cdots \cdots$

{\tt (or (<= n72 n81) (<= n82 n71) (<= m72 m81) (<= m82 m71))}

{\tt ;;not overlap between power components and regular ones}

{\tt (or (<= x12 n11) (<= n12 x11) (<= y12 m11) (<= m12 y11))}

{\tt (or (<= x12 n21) (<= n22 x11) (<= y12 m21) (<= m22 y11))}

$\cdots \cdots$

{\tt (or (<= x32 n81) (<= n82 x31) (<= y32 m81) (<= m82 y31))}

{\tt ;;Constraint 4}

{\tt (or}

{\tt (and (or (= x11 n12) (= x12 n11)) (not (or (< y12 m11) (> y11 m12))))}

{\tt (and (or (= x21 n12) (= x22 n11)) (not (or (< y22 m11) (> y21 m12))))}

$\cdots \cdots$

{\tt (and (or (= y31 m12) (= y32 m11)) (not (or (< x32 n11) (> x31 n12))))}

{\tt )}

$\cdots \cdots$

{\tt (or}

{\tt (and (or (= x11 n82) (= x12 n81)) (not (or (< y12 m81) (> y11 m82))))}

{\tt (and (or (= x21 n82) (= x22 n81)) (not (or (< y22 m81) (> y21 m82))))}

$\cdots \cdots$

{\tt (and (or (= y31 m82) (= y32 m81)) (not (or (< x32 n81) (> x31 n82))))}

{\tt )}

{\tt ;;Constraint 5}

{\tt (or (>= (- (+ x11 (/ (- x12 x11) 2)) (+ x21 (/ (- x22 x21) 2))) 17)}

{\tt (>= (- (+ x21 (/ (- x22 x21) 2)) (+ x11 (/ (- x12 x11) 2))) 17)}

{\tt (>= (- (+ y11 (/ (- y12 y11) 2)) (+ y21 (/ (- y22 y21) 2))) 17)}

{\tt (>= (- (+ y21 (/ (- y22 y21) 2)) (+ y11 (/ (- y12 y11) 2))) 17))}

$\cdots \cdots$

{\tt (or (>= (- (+ x11 (/ (- x12 x11) 2)) (+ x31 (/ (- x32 x31) 2))) 17)}

{\tt (>= (- (+ x31 (/ (- x32 x31) 2)) (+ x11 (/ (- x12 x11) 2))) 17)}

{\tt (>= (- (+ y11 (/ (- y12 y11) 2)) (+ y31 (/ (- y32 y31) 2))) 17)}

{\tt (>= (- (+ y31 (/ (- y32 y31) 2)) (+ y11 (/ (- y12 y11) 2))) 17))}

{\tt ))}
}

Applying {\tt yices-smt -m part$1\_2$.smt}, we test out a satisfiable chip design plan.

{\footnotesize

{\tt sat}

{\tt(= x11 4)}

{\tt(= x12 6)}

{\tt(= y11 2)}

{\tt(= y12 6)}

{\tt(= x21 0)}

{\tt(= x22 4)}

{\tt(= y21 20)}

{\tt(= y22 22)}

{\tt(= x31 21)}

{\tt(= x32 23)}

{\tt(= y31 10)}

{\tt(= y32 14)}

{\tt(= n11 12)}

{\tt(= n12 21)}

{\tt(= m11 10)}

{\tt(= m12 17)}

{\tt(= n21 23)}

{\tt(= n22 29)}

{\tt(= m21 10)}

{\tt(= m22 22)}

{\tt(= n31 4)}

{\tt(= n32 22)}

{\tt(= m31 17)}

{\tt(= m32 22)}

{\tt(= n41 14)}

{\tt(= n42 21)}

{\tt(= m41 0)}

{\tt(= m42 10)}

{\tt(= n51 0)}

{\tt(= n52 4)}

{\tt(= m51 0)}

{\tt(= m52 20)}

{\tt(= n61 5)}

{\tt(= n62 11)}

{\tt(= m61 6)}

{\tt(= m62 16)}

{\tt(= n71 6)}

{\tt(= n72 14)}

{\tt(= m71 0)}

{\tt(= m72 6)}

{\tt(= n81 21)}

{\tt(= n82 29)}

{\tt(= m81 0)}

{\tt(= m82 10)}

}

\begin{center}
\includegraphics[width=0.8\textwidth]{Part1_2_3.png}
\end{center}

\subsection*{Remark:}

\subsection*{Generalization:}

\section*{Problem 3}

Twelve jobs numbered from 1 to 12 have to be executed satisfying the following requirements:
\begin{itemize}
  \item The running time of job $i$ is $i$, for $i = 1, 2, . . . , 12$.
  \item All jobs run without interrupt.
  \item Job 3 may only start if jobs 1 and 2 have been finished.
  \item Job 5 may only start if jobs 3 and 4 have been finished.
  \item Job 7 may only start if jobs 3, 4 and 6 have been finished.
  \item Job 9 may only start if jobs 5 and 8 have been finished.
  \item Job 11 may only start if Job 10 has been finished.
  \item Job 12 may only start if jobs 9 and 11 have been finished.
  \item Jobs 5,7 en 10 require a special equipment of which only one copy is available, so no two of these jobs may run at the same time.
\end{itemize}
Find a solution of this scheduling problem for which the total running time is minimal.

\vspace{4mm}

\subsection*{Solution:}
We generalize this problem for $n$ number of jobs. First we introduce two variables $S_{i}$ and $F_{i}$ for $i = 1,2,3,...,n$.
\begin{itemize}
  \item $S_{i}$ is the starting time of Job $i$. $S_{i} \geq 0$. For example: If Job 1 starts at 0, then $S_{1} = 0$. $S_{i}$ can be any positive integer number if it is satisfied with all requirements.
  \item $F_{i}$ is the finish time of Job $i$.
\end{itemize}

Secondly, we find the clauses which have to be satisfied by all requirements of this problem.

\begin{description}

  \item[Clause 1:] We defined that the starting time of all jobs have to be greater or equal to zero. This is expressed by the formula
      \[ \bigwedge_{i=1}^n (S_{i} \geq 0).\]

  \item[Clause 2:] The running time of job $i$ is $i$, for $i = 1, 2, . . . , n$, and all jobs run without interrupt. So we have
      \[ \bigwedge_{i=1}^n (F_{i} = S_{i} + i).\]

   \item[Clause 3:] The problem specifies that there is a dependency between some jobs, for instance job 3 may only start until jobs 1 and 2 have finished. In order to generalize this and to make a compact formula, we introduce the set of dependencies $\textbf{D}_{i}$ as the set of $finish$ $times$ of the jobs that have to end before starting job $i$, for $i = 1, 2, . . . , n$. So the set of dependencies of job 3 may look like: $\textbf{D}_{3} = \{ F_{1}, F_{2} \}$. Now we can formulate the clause of dependencies between jobs as follow:
       \[ \bigwedge_{i=1}^n \bigwedge_{d \in D_{i}} (S_{i} \geq d).\]

  \item[Clause 4:] The last requirement states that there is an special equipment that can be only used by one job at a time. So the jobs that require this resource cannot work in parallel. To express this we consider the condition that if a job that requires the resource happens before another job that also needs the resource, then the later must only happen if the first job has finished.

      In order to generalize this and formulate a compact expression, we define $\textbf{P}$ as the set of all jobs that share the resource, and define the mappings $f:\textbf{P} \to \textbf{F}$ to map the jobs in $\textbf{P}$ to its respective finish time, and $s:\textbf{P} \to \textbf{S}$ to map the jobs in $\textbf{P}$ to its respective starting time, where $\textbf{F}$ and $\textbf{S}$ are the sets of all finish times and starting times.

      Using all this elements the formula that expresses this clause is the following:

      \[ \bigwedge_{p,q: \; p,q \in \textbf{P} \land p \neq q} (s(p) \leq s(q)) \Rightarrow (f(p) \leq s(q)) .\]

   \item[Clause 5:] The problem also require to find a solution of this scheduling problem for which the total running time is minimal. In order to find such scheduling, we introduce a variable $T$ to represent the allowed running time, now we have to bound the finish time of all jobs to this variable. This can be expressed with the following formula:
       \[ \bigwedge_{i=1}^n (F_{i} \leq T).\]


\end{description}


The total formula now consists of the conjunction of all these clauses.
\[ \bigwedge_{i=1}^n (S_{i} \geq 0)\;\;\wedge\]
\[ \bigwedge_{i=1}^n (F_{i} = S_{i} + i)\;\;\wedge\]
\[ \bigwedge_{i=1}^n \bigwedge_{d \in D_{i}} (S_{i} \geq d)\;\;\wedge\]
\[ \bigwedge_{p,q: \; p,q \in \textbf{P} \land p \neq q} (s(p) \leq s(q)) \Rightarrow (f(p) \leq s(q)) \;\;\wedge\]
\[ \bigwedge_{i=1}^n (F_{i} \leq T).\]


The complete formula expressed in SMT syntax choosing $n=12$, $D_{3} = \{ F_{1}, F_{2} \}$, $D_{5} = \{F_{3}, F_{4}\}$, $D_{7} = \{F_{3}, F_{4}, F_{6}\}$, $D_{9} = \{F_{5}, F_{8}\}$, $D_{11} = \{F_{10}\}$, $D_{12} = \{F_{9}, F_{11}\}$, $P = \{Job\;5, Job\;7, Job\;10\}$ and $T = 36$ is as follow:

{\footnotesize

{\tt (benchmark part1$\_$3}

{\tt :logic $QF\_UFLIA$}

{\tt :extrafuns (}

{\tt (S1 Int) (S2 Int) (S3 Int) (S4 Int)  (S5 Int)  (S6 Int)}

{\tt (S7 Int) (S8 Int) (S9 Int) (S10 Int) (S11 Int) (S12 Int)}

{\tt (F1 Int) (F2 Int) (F3 Int) (F4 Int)  (F5 Int)  (F6 Int)}

{\tt (F7 Int) (F8 Int) (F9 Int) (F10 Int) (F11 Int) (F12 Int))}

{\tt :formula (and}

{\tt (>= S1 0) (>= S2 0) (>= S3 0) (>= S4 0)  (>= S5 0)  (>= S6 0)}

{\tt (>= S7 0) (>= S8 0) (>= S9 0) (>= S10 0) (>= S11 0) (>= S12 0)}

$\cdots \cdots$

{\tt ;All jobs run without interrupt}

{\tt (= (+ S1 1) F1)}

{\tt (= (+ S2 2) F2)}

{\tt (= (+ S3 3) F3)}

$\cdots \cdots$

{\tt ;Job dependencies}

{\tt  (>= S3 F1)  (>= S3 F2)}

{\tt  (>= S5 F3)  (>= S5 F4)}

{\tt (>= S7 F3)  (>= S7 F4) (>= S7 F6)}

$\cdots \cdots$

{\tt ;Jobs 5,7 and 10 cannot execute in parallel}

{\tt (implies (<= S5 S7)   (<= F5 S7))}

{\tt (implies (<= S5 S10)  (<= F5 S10))}

{\tt (implies (<= S7 S5)   (<= F7 S5))}

$\cdots \cdots$

{\tt (<= F1 36)}

{\tt (<= F2 36)}

{\tt (<= F3 36)}

$\cdots \cdots$
}

\vspace{3mm}

Applying {\tt yices-smt -m part$1\_3$.smt} to test out a satisfiable scheduling, we got the following result:

{\footnotesize

{\tt sat}

{\tt (= S1 1)}

{\tt (= S2 0)}

{\tt (= S3 5)}

{\tt (= S4 3)}

$\cdots \cdots$

{\tt (= F5 15)}

{\tt (= F6 6)}

{\tt (= F7 32)}

{\tt (= F8 15)}

{\tt (= F9 24)}

{\tt (= F10 10)}

{\tt (= F11 24)}

{\tt (= F12 36)}
}

\vspace{3mm}

We know that this is the schedule where the total running time is minimal because we started increasing the value of $T$ from 9 and stop until the SMT is SAT. Finally, we find when $T = 35$, it is UNSAT, but when $T = 36$, it is SAT. Therefore, we conclude that the minimal running time satisfying the requirements is $36$.

The following picture illustrates such schedule:

\begin{center}
\includegraphics[width=1.0\textwidth]{Part1_3_1.png}
\end{center}

\subsection*{Remark:}

For this particular problem, the minimum running time was found testing the value of $T$ for different values until we find the minimum satisfiable. Although doing this was relatively easy because the number of jobs is small, and therefore also the minimum running time, this method of finding the minimum value by hand is not very suitable for longer number of jobs and dependencies between them. For instance, if we have $1000$ jobs with their respective dependencies, it would be very time consuming to find such value by hand. Since Yices 1.2 does not have a method for automatically find minimum values, a possible solution is to implement a script that performs the tests automatically.

\subsection*{Generalization:}

We generalized this problem for any number of jobs and dependencies between them. Since we solved this choosing the values that states the problem, it would be interesting  to know the results after changing some parameters. For instance, lets imagine that we have enough copies of the special equipment required by jobs $5$, $7$ and $10$, so they can run in parallel. For this case, the minimum running time is $33$, this is not a big improvement so maybe the money spent by buying the extra equipment does not pay off in performance. 

\section*{Problem 4}

Give a precise description of a non-trivial problem of your own choice, and encode this and solve it by one of the given programs.

\textbf{Self-defined problem:}

In an undirected network the edges are colored red, blue and yellow, and the following is given:

\begin{itemize}
  \item From $A$ there is a red edge to either $C, E$ or $G$.
  \item There are red edges $BF, BI$, and $CH$.
  \item From $G$ there is a yellow edge to either $D$ or $F$.
  \item There is a yellow edge $EG$.
  \item From $D$ there is a blue edge to either $A$ or $B$.
  \item There are blue edges $CG$, $DI$, and $EH$.
\end{itemize}
prove that a path from $A$ to $B$ exists in which no two consecutive edges
are of the same color.

\vspace{4mm}

\subsection*{Solution:}

To solve this problem, we need to specify the definitions for the edges and the paths.

Edge definition: A red edge from $x$ to $y$ is denoted as $red(x,y)$. A yellow edge from $x$ to $y$ is denoted as $yellow(x,y)$. A blue edge from $x$ to $y$ is denoted as $blue(x,y)$. Since it is an undirected network, $red(x,y) = red(y,x)$, $yellow(x,y) = yellow(y,x)$ and $blue(x,y) = blue(y,x)$.

Path definition: A path can be an edge or a sequential connected edges. A path starting with a red edge is called as $redpath$. Similarly, there are $bluepath$ and $ypath$. Particularly, a $redpath$ can be a single red edge or a red edge followed by a $bluepath$ or a $ypath$. Likewise, a $bluepath$ can be a single blue edge or a blue edge followed by a $redpath$ or a $ypath$, and a $ypath$ can be a single yellow edge or a yellow edge followed by a $redpath$ or a $bluepath$. In this way we can make sure that the paths, which are $redpath$, $bluepath$ or $ypath$, will not have two consecutive edges in the same color.

We generalize this problem for an undirected network with $n$ nodes and $m$ colors of the edges. We introduce 
\begin{center}
$\mathbb{N}$ as the set of nodes, $\mathbb{N} = \{A, B, C, ... \}$, 

$\mathbb{E}$ as the set of colors, $\mathbb{E} = \{color_1, color_2, ...\}$, 

and $\mathbb{P}$ as the set of color paths, $\mathbb{P} = \{cpath_1, cpath_2, ...\}$. 
\end{center}
Subsequently, we have the mappings
\begin{center}
$color_x : \mathbb{N}\times \mathbb{N} \rightarrow \mathbb{B}$, and $cpath_x : \mathbb{N}\times \mathbb{N} \rightarrow \mathbb{B}$.
\end{center}

With reference to our definitions, both edges and paths have dependence of the colors. So we introduce $\mathbb{D}_x$ as the set of dependence relations with $color_x$. Now we can formulate the conditions.

\begin{itemize}
  \item Edge definition
  
  Each edge has a starting node and a ending node, and each edge has color dependency.
  \[ \bigwedge_{x=1}^m \forall_{d \in \mathbb{D}} [color_x(fst(d), snd(d))] \;\; \wedge \]
  \[ \bigwedge_{x=1}^m \forall_{n_1, n_2 \in \mathbb{N}} [color_x(n_1, n_2) : color_x(n_2, n_1)] \]
  
  \item Path definition
  
  Likewise, each path has a starting node and a ending node, and each path has color dependency.
  \[ \bigwedge_{x=1}^m \forall_{n_1, n_2 \in \mathbb{N}} [cpath_x(n_1, n_2) : cpath_x(n_2, n_1)] \;\; \wedge \]
  \[ \bigwedge_{x=1}^m \bigwedge_{y=1,y\neq x}^m \forall_{n_1, n_2, n_3 \in \mathbb{N}} [color_x(n_1, n_2) \wedge cpath_y(n_2, n_3) : cpath_x(n_2, n_3)] \]
  
  \item Goals
  
  The goal of the problem is to prove the existence of a path from two nodes, $A$ to $B$, composed with the edges that none of them has the same color as its neighbour(s).
  \[ \bigvee_{x=0}^m cpath_x(A, B) \]
  
\end{itemize}

This problem gives $\mathbb{N} = \{A, B, C, D, E, F, G, H, I \}$ and $\mathbb{E} = \{red, yellow, blue \}$. Accordingly, we derive $\mathbb{P} = \{redpath, ypath, bluepath\}$. We use {\tt Prover9} to prove the existence of such a path. The codes are as follows.

\vspace{3mm}

{\footnotesize

{\tt formulas(assumptions).}

{\tt \% edge definition }

{\tt \% From A there is a red edge to either C, E or G. }

{\tt (red(a,c) \& red(a,e)) | red(a,g).}

{\tt \% There are red edges BF, BI, and CH. }

{\tt red(b,f). red(b,i). red(c,h).}

{\tt \% From G there is a yellow edge to either D or F.}

{\tt yellow(g,d) | yellow(g,f).}

{\tt \% There is a yellow edge EG.}

{\tt yellow(e,h).}

{\tt \% From D there is a blue edge to either A or B.}

{\tt blue(d,a) | blue(d,b).}

{\tt \% There are blue edges CG, GH and DI. }

{\tt blue(c,g). blue(d,i). blue(g,h).}

{\tt \%This is an undirected network.}

{\tt red(x,y) -> red(y,x). yellow(x,y) -> yellow(y,x). blue(x,y) -> blue(y,x).}

{\tt \% path definition}

{\tt red(x,y) -> redpath(x,y).}

{\tt yellow(x,y) -> ypath(x,y).}

{\tt blue(x,y) -> bluepath(x,y).}

{\tt red(x,y) \& bluepath(y,z) -> redpath(x,z).}

{\tt red(x,y) \& ypath(y,z) -> redpath(x,z).}

{\tt yellow(x,y) \& red(y,z) -> ypath(x,z).}

{\tt yellow(x,y) \& bluepath(y,z) -> ypath(x,z).}

{\tt blue(x,y) \& redpath(y,z) -> bluepath(x,z).}

{\tt blue(x,y) \& ypath(y,z) -> bluepath(x,z).}

{\tt redpath(x,y) -> redpath(y,x).}

{\tt bluepath(x,y) -> bluepath(y,x).}

{\tt ypath(x,y) -> ypath(y,x).}

{\tt end\_of\_list.}

{\tt formulas(goals).}

{\tt redpath(a,b) | bluepath(a,b) | ypath(a,b).}

{\tt end\_of\_list.}

}

\vspace{2mm}

After applying {\tt prover9 -f part2\_4.in}, {\tt Prover9} exits with 1 proof. Hence, the existence that a path from $A$ to $B$ has no two consecutive edges in the same color is proved to be true.

\subsection*{Remark:}
We proved that there exits a path from $A$ to $B$ has no two consecutive edges in the same color. If a real applicable solution is provided, the proof will be more reliable. However, {\tt Prover9} is a theorem prover for giving proofs based on resolution. It generates only the proof steps(resolution steps), not a satisfiable solution. Another approach is to derive the solution from the resolution steps. Unfortunately, the proof consists of 406 steps. It is too complicated to process them.  

{\tt Prover9} exits with the following proof.

\vspace{3mm}

{\footnotesize

{\tt ============================== PROOF =================================}

{\tt \% Proof 1 at 0.05 (+ 0.03) seconds.}

{\tt \% Length of proof is 58.}

{\tt \% Level of proof is 11.}

{\tt \% Maximum clause weight is 9.}

{\tt \% Given clauses 322.}

{\tt 1 red(a,c) \& red(a,e) | red(a,g) \# label(non\_clause).  [assumption].}

{\tt 2 red(x,y) -> red(y,x) \# label(non\_clause).  [assumption].}

$\cdots \cdots \cdots$

$\cdots \cdots \cdots$

{\tt 395 yellow(g,d).  [resolve(394,a,196,b),unit\_del(b,44)].}

{\tt 402 -ypath(g,b).  [ur(39,a,394,a,c,44,a)].}

{\tt 406 \$F.  [resolve(395,a,168,a),unit\_del(a,402)].}

{\tt ============================== end of proof ==========================}

}

\vspace{2mm}

\subsection*{Generalization:}
The two introduced sets $\mathbb{N}$ and $\mathbb{E}$ ,are inter-related. The number of nodes in the network limits the number of colors in the network. For instance, there is a two-node network. This network has only two edges, then two colors in maximum.

Once there exists a satisfiable path, there should be infinitely many paths satisfy that the path has no two consecutive edges in the same color. Because the same node may be reached many times. In real-time world, the same actions done in different time unit are different.





\end{document}
