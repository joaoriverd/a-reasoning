\section*{Problem 4}

Give a precise description of a non-trivial problem of your own choice, and encode this and solve it by one of the given programs.

\textbf{Self-defined problem:}

In an undirected network the edges are colored red, blue and yellow, and the following is given:

\begin{itemize}
  \item From $A$ there is a red edge to either $C, E$ or $G$.
  \item There are red edges $BF, BI$, and $CH$.
  \item From $G$ there is a yellow edge to either $D$ or $F$.
  \item There is a yellow edge $EG$.
  \item From $D$ there is a blue edge to either $A$ or $B$.
  \item There are blue edges $CG$, $DI$, and $EH$.
\end{itemize}
prove that a path from $A$ to $B$ exists in which no two consecutive edges
are of the same color.

\vspace{4mm}

\subsection*{Solution:}

To solve this problem, we need to specify the definitions for the edges and the paths.

Edge definition: A red edge from $x$ to $y$ is denoted as $red(x,y)$. A yellow edge from $x$ to $y$ is denoted as $yellow(x,y)$. A blue edge from $x$ to $y$ is denoted as $blue(x,y)$. Since it is an undirected network, $red(x,y) = red(y,x)$, $yellow(x,y) = yellow(y,x)$ and $blue(x,y) = blue(y,x)$.

Path definition: A path can be an edge or a sequential connected edges. A path starting with a red edge is called as $redpath$. Similarly, there are $bluepath$ and $ypath$. Particularly, a $redpath$ can be a single red edge or a red edge followed by a $bluepath$ or a $ypath$. Likewise, a $bluepath$ can be a single blue edge or a blue edge followed by a $redpath$ or a $ypath$, and a $ypath$ can be a single yellow edge or a yellow edge followed by a $redpath$ or a $bluepath$. In this way we can make sure that the paths, which are $redpath$, $bluepath$ or $ypath$, will not have two consecutive edges in the same color.

We generalize this problem for an undirected network with $n$ nodes and $m$ colors of the edges. We introduce
\begin{center}
$\mathbb{N}$ as the set of nodes, $\mathbb{N} = \{A, B, C, ... \}$,

$\mathbb{E}$ as the set of colors, $\mathbb{E} = \{color_1, color_2, ...\}$,

and $\mathbb{P}$ as the set of color paths, $\mathbb{P} = \{cpath_1, cpath_2, ...\}$.
\end{center}
Subsequently, we have the mappings
\begin{center}
$color_x : \mathbb{N}\times \mathbb{N} \rightarrow \mathbb{B}$, and $cpath_x : \mathbb{N}\times \mathbb{N} \rightarrow \mathbb{B}$.
\end{center}

With reference to our definitions, both edges and paths have dependence of the colors. So we introduce $\mathbb{D}_x$ as the set of dependence relations with $color_x$. Now we can formulate the conditions.

\begin{itemize}
  \item Edge definition

  Each edge has a starting node and a ending node, and each edge has color dependency.
  \[ \bigwedge_{x=1}^m \forall_{d \in \mathbb{D}} [color_x(fst(d), snd(d))] \;\; \wedge \]
  \[ \bigwedge_{x=1}^m \forall_{n_1, n_2 \in \mathbb{N}} [color_x(n_1, n_2) : color_x(n_2, n_1)] \]

  \item Path definition

  Likewise, each path has a starting node and a ending node, and each path has color dependency.
  \[ \bigwedge_{x=1}^m \forall_{n_1, n_2 \in \mathbb{N}} [cpath_x(n_1, n_2) : cpath_x(n_2, n_1)] \;\; \wedge \]
  \[ \bigwedge_{x=1}^m \bigwedge_{y=1,y\neq x}^m \forall_{n_1, n_2, n_3 \in \mathbb{N}} [color_x(n_1, n_2) \wedge cpath_y(n_2, n_3) : cpath_x(n_2, n_3)] \]

  \item Goals

  The goal of the problem is to prove the existence of a path from two nodes, $A$ to $B$, composed with the edges that none of them has the same color as its neighbour(s).
  \[ \bigvee_{x=0}^m cpath_x(A, B) \]

\end{itemize}

This problem gives $\mathbb{N} = \{A, B, C, D, E, F, G, H, I \}$ and $\mathbb{E} = \{red, yellow, blue \}$. Accordingly, we derive $\mathbb{P} = \{redpath, ypath, bluepath\}$. We use {\tt Prover9} to prove the existence of such a path. The codes are as follows.

\vspace{3mm}

{\footnotesize

{\tt formulas(assumptions).}

{\tt \% edge definition }

{\tt \% From A there is a red edge to either C, E or G. }

{\tt (red(a,c) \& red(a,e)) | red(a,g).}

{\tt \% There are red edges BF, BI, and CH. }

{\tt red(b,f). red(b,i). red(c,h).}

{\tt \% From G there is a yellow edge to either D or F.}

{\tt yellow(g,d) | yellow(g,f).}

{\tt \% There is a yellow edge EG.}

{\tt yellow(e,h).}

{\tt \% From D there is a blue edge to either A or B.}

{\tt blue(d,a) | blue(d,b).}

{\tt \% There are blue edges CG, GH and DI. }

{\tt blue(c,g). blue(d,i). blue(g,h).}

{\tt \%This is an undirected network.}

{\tt red(x,y) -> red(y,x). yellow(x,y) -> yellow(y,x). blue(x,y) -> blue(y,x).}

{\tt \% path definition}

{\tt red(x,y) -> redpath(x,y).}

{\tt yellow(x,y) -> ypath(x,y).}

{\tt blue(x,y) -> bluepath(x,y).}

{\tt red(x,y) \& bluepath(y,z) -> redpath(x,z).}

{\tt red(x,y) \& ypath(y,z) -> redpath(x,z).}

{\tt yellow(x,y) \& red(y,z) -> ypath(x,z).}

{\tt yellow(x,y) \& bluepath(y,z) -> ypath(x,z).}

{\tt blue(x,y) \& redpath(y,z) -> bluepath(x,z).}

{\tt blue(x,y) \& ypath(y,z) -> bluepath(x,z).}

{\tt redpath(x,y) -> redpath(y,x).}

{\tt bluepath(x,y) -> bluepath(y,x).}

{\tt ypath(x,y) -> ypath(y,x).}

{\tt end\_of\_list.}

{\tt formulas(goals).}

{\tt redpath(a,b) | bluepath(a,b) | ypath(a,b).}

{\tt end\_of\_list.}

}

\vspace{2mm}

After applying {\tt prover9 -f part2\_4.in}, {\tt Prover9} exits with 1 proof. Hence, the existence that a path from $A$ to $B$ has no two consecutive edges in the same color is proved to be true.

\subsection*{Remark:}
We proved that there exits a path from $A$ to $B$ has no two consecutive edges in the same color. If a real applicable solution is provided, the proof will be more reliable. However, {\tt Prover9} is a theorem prover for giving proofs based on resolution. It generates only the proof steps(resolution steps), not a satisfiable solution. Another approach is to derive the solution from the resolution steps. Unfortunately, the proof consists of 406 steps. It is too complicated to process them.

{\tt Prover9} exits with the following proof.

\vspace{3mm}

{\footnotesize

{\tt ============================== PROOF =================================}

{\tt \% Proof 1 at 0.05 (+ 0.03) seconds.}

{\tt \% Length of proof is 58.}

{\tt \% Level of proof is 11.}

{\tt \% Maximum clause weight is 9.}

{\tt \% Given clauses 322.}

{\tt 1 red(a,c) \& red(a,e) | red(a,g) \# label(non\_clause).  [assumption].}

{\tt 2 red(x,y) -> red(y,x) \# label(non\_clause).  [assumption].}

$\cdots \cdots \cdots$

$\cdots \cdots \cdots$

{\tt 395 yellow(g,d).  [resolve(394,a,196,b),unit\_del(b,44)].}

{\tt 402 -ypath(g,b).  [ur(39,a,394,a,c,44,a)].}

{\tt 406 \$F.  [resolve(395,a,168,a),unit\_del(a,402)].}

{\tt ============================== end of proof ==========================}

}

\vspace{2mm}

\subsection*{Generalization:}
The two introduced sets $\mathbb{N}$ and $\mathbb{E}$ ,are inter-related. The number of nodes in the network limits the number of colors in the network. For instance, there is a two-node network. This network has only two edges, then it can have two colors in maximum.

Once there exists a satisfiable path, there should be infinitely many paths satisfying the fact that the path has no two consecutive edges in the same color. Because one node may be reached several times. In the real-time world, the same actions done in different time units lead to different consequences.



