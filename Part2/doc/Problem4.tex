\section*{Problem 4}

Give a precise description of a non-trivial problem of your own choice, and encode this and solve it by one of the given programs.

\subsubsection*{self-defined problem:}

In an undirected network the edges are colored red, blue and yellow, and the following is given:

\begin{itemize}
  \item From $A$ there is a red edge to either $C, E$ or $G$.
  \item There are red edges $BF, BI, CH, DJ, EI, GI$ and $IJ$.
  \item From $G$ there is a yellow edge to either $D$ or $F$.
  \item There is a yellow edge $EG$.
  \item From $D$ there is a blue edge to either $A$ or $B$.
  \item There are blue edges $CG, DI, EH, FJ$ and $GH$.
\end{itemize}
prove that a path from $A$ to $B$ exists in which no two consecutive edges
are of the same color.

\vspace{4mm}

\subsection*{Solution:}

We use $Prover9$ to solve the problem.

Edge definition: A red edge from $x$ to $y$ is denoted as $red(x,y)$. A yellow edge from $x$ to $y$ is denoted as $yellow(x,y)$. A blue edge from $x$ to $y$ is denoted as $blue(x,y)$. Since it is an undirected network, $red(x,y) = red(y,x)$, $yellow(x,y) = yellow(y,x)$ and $blue(x,y) = blue(y,x)$.

Path definition: A path can be an edge or a sequential connected edges. A path starting with a red edge is called as $redpath$. Similarly, there are $bluepath$ and $ypath$. Particularly, a $redpath$ can be a single red edge or a red edge followed by a $bluepath$ or a $ypath$. Likewise, a $bluepath$ can be a single blue edge or a blue edge followed by a $redpath$ or a $ypath$, and a $ypath$ can be a single yellow edge or a yellow edge followed by a $redpath$ or a $bluepath$. In this way we can make sure that the paths, which are $redpath$, $bluepath$ or $ypath$, will not have two consecutive edges in the same color.

Therefore, our $Prover9$ codes are as follows.

{\footnotesize

{\tt formulas(assumptions).}

{\tt \% edge definition }

{\tt red(a,c) | red(a,e) |red(a,g).}

{\tt red(b,f). red(b,i). red(c,h).}

{\tt yellow(g,d) | yellow(g,f).}

{\tt yellow(e,h).}

{\tt blue(d,a) | blue(d,b).}

{\tt blue(c,g). blue(d,i).}

{\tt blue(g,h).}

{\tt red(x,y) -> red(y,x).}

{\tt yellow(x,y) -> yellow(y,x).}

{\tt blue(x,y) -> blue(y,x).}

{\tt \% path definition}

{\tt red(x,y) -> redpath(x,y).}

{\tt yellow(x,y) -> ypath(x,y).}

{\tt blue(x,y) -> bluepath(x,y).}

{\tt red(x,y) \& bluepath(y,z) -> redpath(x,z).}

{\tt red(x,y) \& ypath(y,z) -> redpath(x,z).}

{\tt yellow(x,y) \& red(y,z) -> ypath(x,z).}

{\tt yellow(x,y) \& bluepath(y,z) -> ypath(x,z).}

{\tt blue(x,y) \& redpath(y,z) -> bluepath(x,z).}

{\tt blue(x,y) \& ypath(y,z) -> bluepath(x,z).}

{\tt end\_of\_list.}

{\tt formulas(goals).}

{\tt redpath(a,b) | bluepath(a,b) | ypath(a,b).}

{\tt end\_of\_list.}

}

After applying $Prover9$, the existence that a path from $A$ to $B$ has no two consecutive edges in the same color is proved to be true.

\subsection*{Remark:}


\subsection*{Generalization:}




